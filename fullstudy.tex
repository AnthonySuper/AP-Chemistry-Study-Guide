\documentclass[]{article}
\usepackage{mhchem}
\usepackage{amsmath}
\usepackage{hyperref}
\usepackage{xcolor}
\hypersetup{
    colorlinks,
    linkcolor={red!50!black},
    citecolor={blue!50!black},
    urlcolor={blue!80!black}
}
\begin{document}
\title{Advanced Placement Chemistry Study Guide}
\author{Anthony Super \and Connor Stauder \and Ben Murray}
\maketitle
\tableofcontents
\section{Electron Configuration and Periodic Trends}
\subsection{Orbitals and State}
The ground state configuration of an atom is the configuration with the lowest energy levels. Electrons will eventually settle back to this state if disturbed.

The first two levels of oribtal are S orbitals. There is also a P orbital. The P orbital is in a weird "Squished balloon" type of shape.

\subsection{Atomic Trends}
\begin{description}
\item[Electro-negativity:] increases as you go up and to the right.
\item[Atomic Radius:] Increases as you go down and to the left. The nucleus becomes less strong as you go to the left. The number of orbitals increases as you go down.
\item[Ionization Energy:] Increases up and to the right. 
\end{description}
\section{Oxidization and Reduction}
Oxidization is loss of electrons, reduction is gaining electrons

If something is bonded to itself it has an oxidization state of zero.
\subsection{Assigning Oxidization Numbers}
\begin{enumerate}
\item The oxidization number for an atom in its elemental form is always zero. This is true when:
	\begin{itemize}
	\item Only one kind of atom is present
	\item charge = 0
	\end{itemize}
\item The oxidization number of a monatomic ion is equal to the charge of that ion.
\item The oxidation number of all Group 1A metals is +1
\item The oxidation number of all Group 2A metals is +2
\item Hydrogen (H) has two possible oxidation numbers:
	\begin{itemize}
	\item +1 when bonded to a nonmetal
	\item -1 when bonded to a metal
	\end{itemize}
\item Oxigen has two possible oxidation numbers:
	\begin{itemize}
	\item -1 in peroxides \ce{O_2^{2+}}... very uncommon
	\item -2 in other compounds

	\end{itemize}
\item The oxidation number of flourine (F) is always -1
\item The sum of the oxidation numbers of all atoms or ions in a neutral compound is zero
\item The sum of the oxidation numbers of all atoms in a polyatomic ion is the charge of the polyatomic ion.
\end{enumerate}
\section{Gasses}
\subsection{Partial Pressure}
\[
	\frac{\text{Volume of a component}}{\text{Total volume}}
	=
	\frac{\text{Partial Pressure of a component}}{\text{Total pressure}}
	=
	\frac{\text{Moles of a component}}{\text{Moles of a mixture}}
\]
\section{Stoichiometry}
Stoichiometry refers to the process of determining moles and other measurements of a substance.

\subsection{Balancing}
There must be equal numbers of substances on both sides of a given equation (unless a nuclear reaction is taking place). 

Here is an unbalanced reaction:
\ce{C8H18 + O2 -> CO2 + H2O}.
In order to balance it, there must be an equal number of substances on both sides. For this particular equation, we will handle carbon first:
\ce{C8H18 + O2 -> 8CO2 + H2O}. Now we must balance the hydrogens: \ce{C8H18 + O2 -> 8CO2 + 9H20}. Note how we have excess oxygen. We fix that in the next step, multiplying everything by two in order to get rid of the fraction: \ce{2C8H18 + 9O2  -> 16CO2 + 18H2O}

\subsection{Moles}

There are various forms of a mole:

\begin{tabular}{| c | c | c|}
\hline
Gas at STP & Particles & Grams \\
\hline
22.4 & $ 6.02 \times 10^{23} $ & Molar Mass \\ 
\hline
\end{tabular}

\subsection{Types of Reactions}
\subsubsection{Single Replacement}
Generally ionic, this reaction involves one ion swapping with another. \ce{Cu + AgNO3 -> Ag + Cu(NO3)2}. 
\subsubsection{Double Replacement}
Generally ionic, this reaction involves one ion swapping with another. 
\ce{FeS + 2HCl -> FeCl2 + H2S}
\subsubsection{Synthesis}
Two molecules fusing into one. \ce{2H2 + O2 -> 2H2O}
\subsubsection{Decomposition}
One molecule becoming two or more. 
\ce{2HgO -> 2Hg + O2}
\subsubsection{Combustion}
A special reaction. Some kind of hydrocarbon combines with oxygen to produce a lot of heat, carbon dioxide, and water.
\ce{2C8H18 + 9O2  -> 16CO2 + 18H2O}

\section{Solutions}
Solutions are ions dissolved in a liquid.
\subsection{Definitions}
\begin{description}
\item[Solution:] ions dissolved in a liquid.
\item[Solvent:] the liquid which dissolves the ions.
\item[Solute:] the ions being solved.
\item[Molarity:] $ \frac{\text{Moles}}{\text{Liter}} $
\item[Saturation:] the point where the the solution cannot dissolve more ions
\item[Slightly Soluble:] An ionic compound with dissolves a bit, but not completely.
\item[Soluble:] A compound that dissolves all the way.
\item[Insoluble:] A compound that does not dissolve at all.
\end{description}
\subsection{Solubility Rules}
\begin{tabular}{p{.40\textwidth}  p{.2\textwidth}  p{.40\textwidth}}
\hline
Soluble Ionic Compounds & & Important exceptions \\
\hline
Compounds Containing & \ce{NO3-} & None \\
& \ce{C2H3O2-} & None \\
& \ce{CI-} & None \\
& \ce{Br-} & Compounds of \ce{Ag+}, \ce{Hg^2+}, and \ce{Pb2^2+} \\
& \ce{I-} & Compounds of \ce{Ag+}, \ce{Hg^2+}, and \ce{Pb2^2+} \\
& \ce{SO4--} & Compounds of \ce{Ag+}, \ce{Hg^2+}, and \ce{Pb2^2+} \\
\hline 
\hline
Insoluble Ionic Compounds & & Important Exceptions \\
\hline 
Compounds Containing & \ce{S^2-} & Compounds of \ce{NH4+}, the alkali metal cations, and \ce{Ca^2+}, \ce{Sr2^2+}, and \ce{Ba^2+} \\
& \ce{CO3^2-} & Compounds of \ce{NH4+} and the alkali metal cations. \\
& \ce{PO4^3-} & Compounds of \ce{NH4+} and the alkali metal cations. \\
& \ce{OH-} & Compounds of the alkali metal catoins and \ce{NH4+}, \ce{Ca^2+}, \ce{Sr^2+}, and \ce{Ba^2+}\\
\hline \\
\end{tabular}
\section{Kinetics}
\subsection{Orders of a Reaction}
\begin{description}
\item[0th order:] The rate of a reaction is constant.
\item[1st order:] The rate is equal to the rate constant, multiplied by the concentration of one reactant.
\item[2nd order:] The rate of reaction is equal to the rate constant multiplied by the square of the concentration of one reactant.
\end{description}

To determine what the overall order of a reaction is, look at the exponents of the reactants and add them together.

\section{Equilibrium}
Equilibrium reactions are often solved with \hyperref[sec:ICE]{ICE Tables.} 
Take a look at that section.
\subsection{Definitions}
\begin{description}
\item[Equilibrium Reaction:] A reaction where the products are becoming reactants and the reactants are becoming products. \ce{COO + H2 <=> CO + H2O}
\item[Equilibrium Constant:] Designated as K, it is equal to $ \frac{\text{[Products]}}{\text{[Reactants]}} $. If K is greater than 1, the equilibrium favours the products. If it's less than 1, it favours the reactants. 

\[
	K_c = \frac{\text{[Product]}^{\text{Coefficient}}}{\text{[Reactant]}^\text{Coefficient}}
\]

The equilibrium constant for the reverse reaction will always be $ \text{K}^{-1} $.
\item[Equilibrium]: the state in an equilibrium reaction where the processes are occurring at the same rate.
\end{description}
\section{Energy and Enthalpy}
\subsection{Definitions}
\begin{description}
\item[Enthalpy] a thermodynamic quantity equivalent to the total heat content of a system. It is equal to the internal energy of the system plus the product of pressure and volume.

\item[Specific Heat] The amount of heat required to raise one gram of a substance by 1 degree Kelvin. 
The amount of heat absorbed by a substance, q, is equal to the product of its specific heat (s), its mass, and its temperature change. $ q = s \times m \times \Delta T $. The higher the specific heat, the harder it is to heat up.
\item[Heat Capacity] The amount of heat required to raise a certain amount of a substance by 1 degree Kelvin. Find it with $ \text{specific heat} \times \text{grams of substance} $. If you know how much energy you put into a substance, you can also find the specific heat with $ \frac{\text{joules}}{\text{change in temperature}} $
\item[Joule:] SI unit of work or energy. Equal to the work done by a force of one newton when its point of application moves one meter in the direction of the force.
\item[Hess' Law:] The change in enthalpy of a chemical reaction is the same regardless of if it is done in one or several steps. Technically an expression of the conservation of energy.
\end{description}
\subsection{Types of Reaction}
An \textbf{endothermic} reaction is one in which $ \Delta H $ is positive. The $ \Delta T $ will be positive. The system absorbs energy. An \textbf{exothermic} reaction in which $ \Delta H $ is negative. The $ \Delta T $ is also negative. The system releases heat into the surroundings.

It takes energy to break chemical bonds. At some point, it will take more energy to break the bonds than the energy the bonds will release when broken. 
\section{Acids and Bases}
\subsection{Definitions}
\begin{description}
\item[Br\o nsted-Lowry Acids:] substances with dissociate in an aqueous solution to give \ce{H+} ions
\item[Br\o nstead-Lowry Bases:] substances with dissociate in an aqueous solution to receive \ce{H+} ions
\item[Arrhenius Acid:] increases \ce{H+} concentration
\item[Arrhenius Base:] increases \ce{OH-} concentration
\item[Neutral Solution:] when the concentration of \ce{H+} is equal to the concentration of \ce{OH-}
\end{description}
\subsection{Useful Equations}
\[
	\text{Ka} = \frac{\text{Concentration of Products}}{\text{Concentration of Reactants}}
\]

\[
	\text{pH} = -\log{ \text{[ \ce{H+}  ]} }
\]

\[
	\text{pOH} = -\log{ \text{[ \ce{OH-}  ]} }
\]

\[
	14 = \text{pH} + \text{pOH}
\]

\[
	\text{pH} = \text{pKa} - \log{ \frac{\text{[Ha]}}{\text{[A-]}}}
\]
\subsection{Strong Acids and Bases}
\begin{tabular}{| p{.5\linewidth} | p{.5\linewidth} |}
\hline
Strong Acids & Strong Bases \\
\hline
\ce{HI} & \ce{NaOH} \\
\hline
\ce{HBr} & \ce{KOH} \\
\hline
\ce{HClO4} & \ce{LiOH} \\
\hline
\ce{HCl} & \ce{RbOH} \\
\hline
\ce{HClO3} & \ce{CsOh} \\
\hline
\ce{H2SO4} & \ce{Ca(OH)2} \\
\hline
& \ce{Ba(OH)2} \\
\hline
& \ce{Sr(OH)2} \\
\hline
\end{tabular}
\subsection{Conjugate Acids and Bases}
A \textbf{Conjugate Acid} or \textbf{Conjugate Base} is the salt of the acid or base. For example, the conjugate base of  \ce{H2O} is \ce{OH-}, and the conjugate base of \ce{CH3COOH} can be \ce{CH3OONA}. The important part in the preceding example is the \ce{CH3OO}. Any additional ion will still make it a conjugate base.

In essence, an acid plus its conjugate base or a base plus its conjugate acid will make a buffer. 
\subsubsection{Buffer}
A \textbf{buffer} is a solution which resists change in pH. A buffer is able to resist pH change because the two components (conjugate acids and conjugate base) are present in the solution in appreciable amounts. The conjugate base neutralizes any added acid, and the conjugate acid neutralizes any base. Adding more water allows more of the acid or base to dissolve, causing the pH to remain the same. Removing water has the opposite effect, for a net equal pH.
\subsection{ICE Table}
\label{sec:ICE}
Example: What is the pH of a 0.020 M Solution of Hydrosulfuric Acid, a diprotic acid. Remember that this acid releases two protons.
\begin{enumerate}
\item 
\begin{tabular}{|c|c|c|c|}
\hline
Substance & \ce{H2S} & \ce{H+} & \ce{HS-} \\
\hline
Initial & 0.002 M & 0 & 0 \\
\hline
Change & -x & +x & + x \\
\hline
Equilibrium & $ 0.020 - x $ & x & x \\
\hline
\end{tabular}
\item
\[
	\text{Ka} = \frac{\text{[\ce{H+}][\ce{HS-}]}}{\text{[\ce{H2S}]}}
\]
\item
\[
	\frac{x^2}{0.02}  = 11\times 10^{-7}
\]
\item
\[
	x^2 = (0.02)(1.1\times 10^{-7})
\]
\item

	 \[ \text{[\ce{H+}]} = x = \sqrt{2.2 \times 10^{-9} } = 4.69 \times 10^{-5} \]
	
\item
\[
	\text{pH} = -\log{\text{[\ce{H+}]}} = -\log{4.69 \times 10^{-5}} = 4.3287
\]

\end{enumerate}
\subsection{Titration}
A \textbf{Titration} is a way to determine the concentration of an unknown acid or base. Simply add a known base (or acid, if you are titrating a base) and graph the pH values as you add it. Eventually, the graph's slope will become very steep. The center of this slope steepness is the point of equilibrium. For strong acids or bases, this equilibrium will be 7. A weak acid or base will be above or below 7, respectively. 
\section{Entropy and Spontaneous Reactions}
\subsection{The laws of Thermodynamics}
\subsubsection{The First Law}
"When energy passes, as work, as heat, or with matter, into or out from a system, its internal energy changes in accord with the law of conservation of energy."

Essentially, energy cannot be created or destroyed. 
\subsubsection{The Second Law}
"In a natural thermodynamic process, the sum of the entropies of the participating thermodynamic systems increases."

This means that anything you do increases entropy.
That is, you make the universe more chaotic with every action.
\subsubsection{The Third Law}
"The entropy of a system approaches a constant value as the temperature approaches absolute zero. With the exception of glasses the entropy of a system at absolute zero is typically close to zero, and is equal to the log of the multiplicity of the quantum ground state."

Essentially, "the entropy of a perfect crystal at absolute zero is zero." 
\subsection{Definitions}
\begin{description}
\item[Entropy:] the measure of chaos in the universe. Some reactions may make the entropy in a system decrease, but every reaction increases the total entropy of the universe. 
\item[Free Energy:] the capacity of a system to do work. When the change in free energy is negative, the reaction is spontaneous. 
\end{description}
\subsection{Spontaneous Reaction Table}
\begin{tabular}{p{.1\linewidth} p{.1\linewidth} p{.2\linewidth} p{.6\linewidth}}
$ \Delta \text{H} $ & $ \Delta \text{S} $ & $ \Delta \text{G}  (\Delta \text{H} - \Delta \text{S})$ & Reaction Spontaneity \\
\hline
$ - $ & $ + $ & $ - $ & Spontaneous at all temperatures \\
$ - $ & $ - $ & $ - $ or $ + $ & Spontaneous at temperatures where $ \Delta \text{H} $ outweighs $ \text{T} \Delta \text{S} $ (low temperatures) \\
$ + $ & $ - $ & $ + $ & Non-spontaneous at all temperatures \\

$ + $ & $ + $ & $ + $ or $ - $ & Spontaneous at temperatures where $ \Delta \text{H} $ outweighs $ \text{T} \Delta \text{S} $ (High temperatures) \\
\end{tabular}
\subsection{Reactions that Increase Entropy}
\begin{itemize}
\item Moving a substance to a higher state. Things like ice melting or water boiling.
\item Fewer moles of a gas becoming more moles. Decomposition reaction.
\item Volume increasing (more microstates).
\end{itemize}
The converse of these reactions will lower entropy.
\end{document}